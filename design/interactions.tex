We provide a preliminary discussion of the interactions of Peras with Mithril\footnote{\label{footnote:mithril}\url{https://mithril.network/}} and Ouroboros Leios\footnote{\label{footnote:leios}\url{https://leios.cardano-scaling.org/}}.

\section{Mithril}\label{sec:mithril}

Mithril is a protocol for generating stake-based threshold multi-signatures for certain entities related to the Cardano blockchain, for example the stake distribution, the node database or even individual transactions.

\subsection{Mithril certificates for more recent state}

By design, Mithril can only sign data that all honest nodes agree on.
Naturally, the faster finality guarantees by Peras therefore allow Mithril to sign data that is more recent.
For example, Mithril-generated transaction inclusion proofs can arise more quickly under optimistic conditions, improving the user experience of applications integrating them.

\subsection{Bootstrapping Peras certificates}
Mithril\footref{footnote:mithril} is often used to bootstrap a node by fetching a pre-synced chain database which has been signed via Mithril instead of syncing manually from other nodes.
Given that Peras is adding \emph{certificates} to the chain database, we now sketch how these could be handled by Mithril.
\begin{itemize}
\item
  Applications that are not interested to serve the chain to other nodes themselves\footnote{
    We note that such nodes might also benefit from other optimizations, such as eliding the transaction signatures from blocks (\enquote{segregated witness}).}
  do not need Peras certificates, so Mithril does already work for this use case without any additional modifications.
\item
  Nodes that \emph{do} want/need to serve the historical chain to other nodes (e.g., pools) need to obtain all historical certificates.
  This is non-trivial, as there can be many different valid certificates for the same block in some round, and even further, honest nodes do not necessarily have the exact same collection of historical certificates, see~\cref{sec:certificate generalities}.
  This is in contrast to the immutable chain, on which all honest peers agree on eventually up to a specific slot.

  This might not be a concern if all certificates are guaranteed to (eventually) end up on-chain, for example as part of a reward system, cf.\ \cref{chap:rewards}.
\end{itemize}

One approach is to use Mithril to first sync the historical ImmutableDB, and then sync the historical certificates separately.
For this, Mithril can provide the historical stake distributions, protocol parameters and epoch nonces, allowing to validate historical certificates without having to maintain an appropriate ledger state.
Such a special \enquote{certificate sync} could be highly optimized (in particular leveraging a high degree of concurrency), still being much faster than a regular full sync without Mithril.

In its simplest form, this extra step of syncing historical certificates must take place before the node is started.
As a further optimization, the node could be started immediately, with historical certificates being synchronized in the background.
In this stage, the node enjoys all security properties of a caught-up node; however, it can not serve as an honest upstream peer for \emph{syncing} nodes, not counting towards the Honest Availability Assumption of Ouroboros Genesis \parencite{genesis-implementation-documentation}.
A potential mitigation is to fully disallow connections from evidently syncing peers, or to only negotiate intersections before the tip of the CertDB, implicitly relying on the fact that no node can sync the chain faster than the serving node can sync just the historical certificates.

Finally, an even more efficient approach might consist in defining a succinct representation of historically boosted blocks which can be signed and distributed via Mithril, but can be validated and consumed even by nodes not syncing via Mithril.
However, this is only a rough sketch, and significant additional work to determine its feasibility would be required.

\subsection{Reusing registered keys}\label{sec:mithril key reuse}

In its current form, Mithril handles registration of new participants off-chain, not being able to benefit from the automatic handling of conflicting registrations enabled by the on-chain consensus.
Plausibly, Peras and Mithril could share the underlying cryptographic scheme used for votes and certificates, enabling a simplification of Mithril at no extra cost.

The current design for votes and certificates in Peras \parencite{peras-cert-report} is different from Mithril's certificates \parencite{chaidos2024mithril}; however, we understand that there is an effort to use a unified approach of stake-based threshold multi-signatures across various Cardano-related projects, in particular for Peras and Mithril.

\subsection{Reusing votes across Peras and Mithril}\label{sec:mithril reuse votes}
Just like Mithril, Peras needs to diffuse votes in order to create certificates, which is a stake-based threshold multi-signature for the block that is being voted for.%
\footnote{The Mithril team intends to use a protocol very similar to Tx-Submission, see~\cite{dmq-cip}.}
Conceptually, it is therefore possible to use the same votes for Peras \emph{and} for Mithril, adding the data to be signed for both applications to individual votes (concretely, the point of the block that Peras is voting for, and e.g.\ the hash of a (stable) stake distribution for Mithril).
Peras and Mithril would aggregate these votes into certificates independently from each other.

However, the possibility and/or efficiency depends on the details of the chosen scheme for votes and certificates; in general, we might need to aggregate signatures for (partially) different messages, which is an unusual requirement.
For example, it is not (directly) supported by~\cite{peras-cert-report}.

The main benefit of this approach is that all pools would automatically and at essentially no extra cost for themselves on top of Peras (assuming that the data that Mithril requires to be signed is sufficiently lightweight, which should be the case for e.g.\ stake distributions) participate in Mithril voting.
In particular, no bandwidth is wasted for duplicate vote diffusion.

Of course, this approach requires that Peras and Mithril use the same or compatible cryptography for votes and certificates as already mentioned in \cref{sec:mithril key reuse}.

Another complication is the existence of Peras cooldown periods.
As presented in the Peras CIP \parencite{peras-cip} and this document, honest nodes stop voting during such a cooldown period, which, even if rare, would also disable Mithril as a side effect, which is undesirable.
A natural mitigation is to let nodes still vote even during Peras cooldown periods, but indicate in the vote that it must not be considered as a Peras vote, only as a Mithril vote.

In general, the main drawback of this idea is that it introduces coupling between components that are logically separate, and hence imposes accidental complexity that needs to be weighed against the aforementioned advantages.
In particular, this coupling makes a separate evolution of Peras and Mithril more difficult.

\section{Ouroboros Leios}\label{sec:leios}

Ouroboros Leios\footref{footnote:leios} is a high-throughput protocol for Cardano.

\subsection{Combining Peras and Leios}

From a very high level point of view, Peras and Leios are orthogonal features:
Leios is a significant change to the protocol with many new kinds of blocks as well as votes and certificates.
However, in the end, Leios still establishes a chain made out of \emph{ranking blocks} that follows the longest chain rule as in Praos.
Peras can still be applied to this chain (i.e.\ votes would be cast for ranking blocks), providing faster settlement under optimistic conditions.

In other words, interacting with the Cardano blockchain via a transaction involves two steps:
\begin{enumerate}
\item\label{enumi:leios:tx step 1}
  First, the transaction needs to be propagated through the network and be included in a block.
\item\label{enumi:leios:tx step 2}
  Second, the block needs to be settled with a probability appropriate for the concrete use case.
\end{enumerate}
Leios accelerates step~\ref{enumi:leios:tx step 1} (by increasing throughput), while Peras speeds up step~\ref{enumi:leios:tx step 2} (by boosting the block or one of its descendants).

On a lower level, Peras and Leios fundamentally compete for bandwidth in the system (as Leios is all about making full use of it, in contrast to Praos).
The Leios innovation team is currently working on elaborate realistic simulations of Leios to assess its dynamics and performance profiles.
Enriching these simulations with Peras (concretely, its vote diffusion) can provide further insights into this tradeoff.
However, we note that a future version of Peras might incorporate a dedicated \emph{pre-agreement} mechanism of unclear specifics.

Overall, there is no explicit dependency between Peras and Leios, but rather common dependencies like the work on a cryptographic scheme for votes and certificates.
Moreover, Leios will likely also use a variant of Tx-Submission to diffuse its various kinds of new entities, which provides an opportunity for collaboration.

\subsection{Reusing votes}
Like Peras and Mithril, Leios also makes use of stake-based threshold multi-signatures (in order to certify data availability), and needs to diffuse votes for this purpose.
Similar to the discussion in~\cref{sec:mithril reuse votes}, it is therefore in principle conceivable to share these votes between Peras and Leios (and Mithril), potentially mitigating the competition for bandwidth.

\medskip
At present, it is unclear whether or how Peras could be more tightly integrated with Leios, such as voting for input blocks or directly reusing the endorsement certificates for faster finality.
It seems likely that any such approach would have to differ significantly from Peras in its current form.


%%% Local Variables:
%%% mode: latex
%%% TeX-engine: xetex
%%% TeX-master: "peras-design"
%%% End:
