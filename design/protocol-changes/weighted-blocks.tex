\section{Using chain weight instead of chain length}

Any certified block receives extra weight corresponding to \perasBoost{} per certificate.%
\footnote{Usually, blocks will be boosted by at most one certificate, but it is possible for the same block to be boosted by multiple certificates, for example if there is no active slot within a Peras round, which happens with probability ${(1-\asc)}^{\perasRoundSlots} \approx \qty{1}{\percent}$ for $\asc=1/20$ and $\perasRoundSlots=90$.}
Various components that currently consider chain \emph{length} must instead consider chain \emph{weight}.

\begin{description}
\item[Chain selection]
  Most prominently, chain selection must no longer select the longest, but instead the \emph{weightiest} chain, using all currently available certificates.
  In particular, the selection of a node can now additionally change solely due to the emergence of a new certificate, without any new blocks.
\item[Volatile vs.\ immutable chain]
  The node maintains its selection as an immutable prefix and a volatile suffix, where only the latter is subject to rollbacks.
  Without Peras, the volatile suffix is defined to be the \kcp{} most recent blocks.
  With Peras, it is instead defined to be the largest suffix containing blocks of weight at least \kcp{}.\footnote{
    Note that it is acceptable (but overly conservative) to not capitalize on this observation, and still consider the last \kcp{} blocks to be volatile.}

  When Peras is not in a cooldown phase, this means that the length of the volatile suffix decreases from \kcp{} blocks to
  \[ \left\lceil \frac{\kcp }{1 + \frac{\perasBoost}{\perasRoundSlots \cdot \asc}} \right\rceil = 499 \]
  blocks in expectation, assuming no adversary and full honest participation, as well as $\perasBoost = 15$, $\asc = 1/20$ and $\kcp = 2160$.

  As a consequence, and because we only need to store ledger states for each volatile block and the immutable tip, fewer ledger states need to be stored in that case.
  However, this expected to only result in a small gain in efficiency, as ledger states corresponding to subsequent blocks only differ by a small amount in relation to the total size, and existing implementations exploit this fact by using implicit structural sharing or by explicitly maintaining these diffs.
\item[Ouroboros Genesis rule]
  The Cardano Genesis implementation \parencite{genesis-implementation-documentation} performs density comparisons between competing header chains while syncing.
  With Peras, this needs to be modified to use the \emph{weighted} density instead (cf.~\cref{sec:weighted genesis}).
\item[BlockFetch decision logic]
  The BlockFetch decision logic is responsible for downloading blocks corresponding to candidate header chains that are preferable to the current selection.
  The comparison between candidate header chains and the current selection needs to take weight into account.
\item[Hard Fork Combinator]
  The Hard Fork Combinator (HFC) is Cardano's mechanism for transparently handling transitions between different Cardano eras.
  Before enacting an era transition decided by on-chain governance, the HFC requires at least \kcp{} blocks after the \emph{voting deadline}, i.e.\ the last slot that could still affect the on-chain governance.\footnote{
    See \url{https://ouroboros-consensus.cardano.intersectmbo.org/docs/for-developers/CivicTime} for more context.}
  This mechanism is called \emph{block counting}.

  Relying on Praos Chain Growth, these \kcp{} blocks must arise in \Tcp{} slots.
  In Peras however, Chain Growth guarantees instead that \Tcp{} slots contain blocks having \emph{weight} of at least \kcp{} blocks on any chain, which does not imply that there actually are \kcp{} blocks.
  See \cref{sec:density reduction via boost-induced rollbacks} for more details on how the adversary can cause the chain to be of high weight, but low (unweighted) density.

  It is conceivable that a more fine-grained analysis could result in the probability to have less than \kcp{} blocks (or an appropriately increased quantity) to still be acceptably low for this purpose, such that the HFC can continue to use block counting.
  We also not that block counting has been under closer scrutiny and been subject of discussions to potentially replace it for other, unrelated reasons.
\end{description}


%%% Local Variables:
%%% mode: latex
%%% TeX-engine: xetex
%%% TeX-master: "../peras-design"
%%% End:
