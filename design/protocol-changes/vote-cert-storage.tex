\section{Vote storage (PerasVoteDB)}\label{sec:vote db}

The node needs to store votes for the current round (cf.~\cref{sec:votes certs from the past}) in order to observe a quorum, and to diffuse these votes to downstream peers while the round is still ongoing even after a quorum has been observed.

\begin{itemize}
\item
  As soon as the PerasVoteDB contains votes with weight of at least \perasQuorum, it starts aggregating these votes into a certificate and adds it to the PerasCertDB (\cref{sec:cert db}).
\item
  At the beginning of a Peras round $r$, all votes prior to $r$ are deleted from the PerasVoteDB\@.
\end{itemize}

We propose to store these votes in memory only; not persisting its state to disk.
Restarting a node is realistically going to take longer than \perasRoundSlots{}, so any persisted votes would become stale immediately after a restart.

\section{Certificate storage (PerasCertDB)}\label{sec:cert db}

\subsection{In-memory storage of recent certificates}
Various components of the node need quick access to all certificates applying to a particular candidate chain in order to compute its weight, cf.~\cref{sec:weight not length}.
In addition, the Peras aspect of the block minting logic (\cref{sec:vote mint}) needs access to the most recent certificate.

We propose to maintain an in-memory cache in the PerasCertDB of all certificates that can apply to a block that the node could select.
Older certificates (e.g.\ those with a round that ends before the tip slot of the immutable chain) can be garbage-collected from this in-memory cache, which bounds the size of the cache.
Therefore, as there is at most one certificate per round, this cache contains at most $\lceil \Tcp / \perasRoundSlots \rceil = 1440$ certificates, so in total \qty{28.8}{\mega\byte} for the proposed parameters (\cref{sec:protocol parameters}).

\subsection{On-disk storage of historical certificates}
Nodes serving the historical chain need to retain all past certificates in order to provide them to peers syncing from them via Ouroboros Genesis, see \cref{sec:weighted genesis}.
Naturally, due to its unbounded size (linear to the age of the chain), these certificates need to be stored on disk.

We store certificates on disk that are boosting a block on the immutable chain.%
\footnote{The only certificates that do not have this property must be from rounds shortly before a cooldown period, so they are rather rare.}
Therefore, the on-disk store is \emph{append-only}, and all certificates stored in it are \emph{immutable}.

These properties are similar to those of the ImmutableDB in the \texttt{cardano-node} implementation, \parencite[chapter~8]{consensus-storage-report}, and the on-disk storage has similar requirements:
\begin{enumerate}
\item
  The store must efficiently add new certificates, but it \emph{is} acceptable if recently added certificates are lost on a crash, as they can be re-downloaded.
\item
  The store must provide performant functionality to implement the server side of certificate diffusion \cref{sec:certificate-diffusion}.
  Concretely, we require efficient lookups and streaming of certificates indexed by their round number.
\end{enumerate}
A simplified variant of the ImmutableDB as implemented in the \texttt{cardano-node} fulfills these requirements, as does any standard key-value store (which usually provide many additional features/guarantees that are not strictly required here).

%%% Local Variables:
%%% mode: latex
%%% TeX-engine: xetex
%%% TeX-master: "../peras-design"
%%% End:
