\section{Votes and certificates}

Peras introduces two new entities: \emph{votes} and \emph{certificates}.

\subsection{Votes}

Peras voting proceeds in rounds, lasting $\perasRoundSlots$ each.
Every round has an associated \emph{weighted committee} \parencite[Section~4]{gavzi2023fait}, a set of stake pools responsible for voting in this round, together with individual amounts of \emph{voting power}, i.e.\ their \emph{weight} in the committee.
The exact committee election scheme has not yet been decided, but it must satisfy the following informal properties.
\begin{enumerate}
\item
  An adversary with total stake $\alpha$ can not get significantly more than $\alpha$ relative weight in any committee.
  The (expected) weight of a pool in the committee is proportional to its stake.
\item
  The (expected) size of the committee can be controlled via the \perasN{} parameter.
\end{enumerate}

At the start of every Peras round (lasting $\perasRoundSlots$), all of the honest nodes in the corresponding Peras committee will usually vote for a recent block (otherwise, Peras is likely entering a cooldown period).
Votes are \emph{weighted} according to the weight of the pool that cast them in the committee, which allows for an improved size-security tradeoff compared to unweighted votes (where the same party might be allowed to vote multiple times), cf.~\cite{gavzi2023fait}.
Under optimistic conditions, votes of total weight of at least \perasQuorum{} are cast for the same block, in which case the round is \emph{successful}.

A vote contains the following data:
\begin{itemize}
\item Round number.
\item Point of the block that is being voted for.
\item Stake pool identifier.
\item Associated cryptographic material, i.e.\ a signature and potentially an eligibility proof.
\end{itemize}
The weight of a vote is implicit here and can be recomputed upon validation.

A vote from round $r$ (starting in slot $s$) for a block in slot $t$ is \emph{valid} under the following conditions.
\begin{itemize}
\item
  The cryptographic material must be valid, ensuring that the vote was indeed cast by an eligible pool.

  This requires the stake distribution, certain protocol parameters and the epoch nonce.
  We suggest using the data that would be used to validate a header for slot $s$.
  \alex{Appendix for forecasting caveat.}
\item
  $s - \perasBlockMaxSlots \le t < s - \perasBlockMinSlots$.
\item
  The slots $s$ and $t$ are in the same era, or at least, the eras of $s$ and $t$ both support Peras.
\end{itemize}

\subsection{Certificates}

Once a node observes votes of the same round for the same block with total weight of at least $\perasQuorum$, then it will create (via \emph{aggregation}) a corresponding (succinct) \emph{certificate} proving this fact.

A certificate contains the following data:
\begin{itemize}
\item Round number.
\item Point of the block that is being voted for.
\item Associated cryptographic material certifying that there are votes of weight \perasQuorum{} voting for the aforementioned block in the given round.
\end{itemize}

Validity of certificates is analogous to that of votes.

We note that the cryptographic material may vary depending on which votes are aggregated.
Such certificates are only artificially different and should be treated interchangeably.

On the other hand, we assume that \emph{equivocating} certificates, i.e.\ two certificates in the same round for \emph{different} blocks, do not occur.
\begin{tcolorbox}[title=Assumption]
  For every Peras round, there is at most one block that has a valid certificate in that round.
\end{tcolorbox}
This assumption is justified by an appropriate parameterization \perasQuorum{} and \perasN{}, such that an adversary never attains sufficient weight in the committee to equivocate certificates.\footnote{Future work on the Peras protocol might relax this assumption and hence allow a less conservative parameterization.}

\subsection{Realizing votes and certificates}

The underlying cryptography used to realized votes and certificates is out-of-scope for this document, we refer to~\cite{peras-cert-report} for more details.
Here, we only summarize a few key metrics in \cref{fig:vote cert metrics}, while noting that these are still subject to change\footnote{In particular, certificates might get significantly smaller ($\le \qty{1}{\kilo\byte}$).}; however, we do not expect them to get significantly worse.

\begin{figure}[h]
  \centering
  \begin{tabular}{c c c}
    \toprule
    & Votes & Certificates \\
    \midrule
    Size & \qtyrange{90}{164}{\byte} & \qtyrange{7}{10}{\kilo\byte} \\
    Generation & \qty{280}{\us} & \qtyrange{63}{93}{\ms} \\
    Verification & \qty{1.4}{\ms} & \qtyrange{115}{157}{\ms} \\
    \bottomrule
  \end{tabular}
  \caption{Preliminary metrics for votes and certificates}\label{fig:vote cert metrics}
\end{figure}

The existing formal specification in Agda for the Consensus and Ledger layer \parencite{consensus-spec,cardano-formal-ledger-specs} can be naturally extended with the details of vote and certificate validation, in particular for conformance testing.

\subsection{Handling votes and certificates from the future}

A vote or certificate is \emph{from the future} if its round has not yet begun according to the local wallclock.
We suggest to handle such votes just like \emph{headers} from the future are already being handled:
\begin{itemize}
\item
  Upon receiving a vote or certificate $x$ from the \emph{near} future (which is currently defined to be at most \qty{2}{\s} in the future), an artifical delay is introduced such that $x$ is no longer from the future, after which $x$ is processed further.

  This rule is motivated by the insight that it would be unreasonable to expect clocks between honest nodes across the world to be perfectly synchronized.

  The artifical delay might not be necessary for Peras votes/certificates as there is, in contrast to headers/blocks, no incentive to diffuse your vote as early as possible (as long as it still arrives within a Peras round).
  However, using the exact same logic as for headers has a conceptual simplicity.
\item
  Upon receiving a vote or certificate from the \emph{far} future (defined to be more than \qty{2}{\s} in the future), we immediately disconnect from the sending peer as this constitutes adversarial behavior.
\end{itemize}


%%% Local Variables:
%%% mode: latex
%%% TeX-engine: xetex
%%% TeX-master: "../peras-design"
%%% End:
