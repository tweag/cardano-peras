\section{Votes and certificates}

Peras introduces two new entities: \emph{votes} and \emph{certificates}.

\subsection{Votes}

Peras voting proceeds in rounds, lasting $\perasRoundSlots$ each.
Every round has an associated \emph{weighted committee} \parencite[Section~4]{gavzi2023fait}, a set of stake pools responsible for voting in this round, together with individual amounts of \emph{voting power}, i.e.\ their \emph{weight} in the committee.
The exact committee election scheme has not yet been decided, but it must satisfy the following informal properties.
\begin{enumerate}
\item
  An adversary with total stake $\alpha$ can not get significantly more than $\alpha$ relative weight in any committee.
  The (expected) weight of a pool in the committee is proportional to its stake.
\item
  The (expected) size of the committee can be controlled via the \perasN{} parameter.
\end{enumerate}

At the start of every Peras round (lasting $\perasRoundSlots$), all of the honest nodes in the corresponding Peras committee will usually vote for a recent block (otherwise, Peras is likely entering a cooldown period).
Votes are \emph{weighted} according to the weight of the pool that cast them in the committee, which allows for an improved size-security tradeoff compared to unweighted votes (where the same party might be allowed to vote multiple times), cf.~\cite{gavzi2023fait}.
Under optimistic conditions, votes of total weight of at least \perasQuorum{} are cast for the same block, in which case the round is \emph{successful}.

A vote contains the following data:
\begin{itemize}
\item Round number.
\item Point of the block that is being voted for.
\item Stake pool identifier.
\item Associated cryptographic material, i.e.\ a signature and potentially an eligibility proof.
\end{itemize}
The weight of a vote is implicit here and can be recomputed upon validation.

A vote from round $r$ (starting in slot $s$) for a block in slot $t$ is \emph{valid} under the following conditions.
\begin{itemize}
\item
  The cryptographic material must be valid, ensuring that the vote was indeed cast by an eligible pool.

  This requires the stake distribution, certain protocol parameters and the epoch nonce.
  We suggest using the data that would be used to validate a header for slot $s$.
  \alex{Appendix for forecasting caveat.}
\item
  $s - \perasBlockMaxSlots \le t < s - \perasBlockMinSlots$.
\item
  The slots $s$ and $t$ are in the same era, or at least, the eras of $s$ and $t$ both support Peras.
\end{itemize}

\subsection{Certificates}\label{sec:certificate generalities}

Once a node observes votes of the same round for the same block with total weight of at least $\perasQuorum$, then it will create (via \emph{aggregation}) a corresponding (succinct) \emph{certificate} proving this fact.

A certificate contains the following data:
\begin{itemize}
\item Round number.
\item Point of the block that is being voted for.
\item Associated cryptographic material certifying that there are votes of weight \perasQuorum{} voting for the aforementioned block in the given round.
\end{itemize}

Validity of certificates is analogous to that of votes.

We note that the cryptographic material may vary depending on which votes are aggregated.
Such certificates are only artificially different and should be treated interchangeably.

On the other hand, we assume that \emph{equivocating} certificates, i.e.\ two certificates in the same round for \emph{different} blocks, do not occur.
\begin{tcolorbox}[title=Assumption]
  For every Peras round, there is at most one block that has a valid certificate in that round.
\end{tcolorbox}
This assumption is justified by an appropriate parameterization of \perasQuorum{} and \perasN{}, such that an adversary never attains sufficient weight in the committee to equivocate certificates.\footnote{Future work on the Peras protocol might relax this assumption and hence allow a less conservative parameterization.
  However, certain design decisions (especially around certificate diffusion) in this document would need to be reconsidered to properly handle equivocating certificates, as we decided not to prematurely complicate the design without a clear picture of how the necessary semantics.}

\subsection{Realizing votes and certificates}

The underlying cryptography used to realized votes and certificates is out-of-scope for this document, we refer to~\cite{peras-cert-report} for more details.
Here, we only summarize a few key metrics in \cref{fig:vote cert metrics}, while noting that these are still subject to change\footnote{In particular, certificates might get significantly smaller ($\le \qty{1}{\kilo\byte}$).}; however, we do not expect them to get significantly worse.

\begin{figure}[h]
  \centering
  \begin{tabular}{c c c}
    \toprule
    & Votes & Certificates \\
    \midrule
    Size & \qtyrange{90}{164}{\byte} & \qtyrange{7}{10}{\kilo\byte} \\
    Generation & \qty{280}{\us} & \qtyrange{63}{93}{\ms} \\
    Verification & \qty{1.4}{\ms} & \qtyrange{115}{157}{\ms} \\
    \bottomrule
  \end{tabular}
  \caption{Preliminary metrics for votes and certificates}\label{fig:vote cert metrics}
\end{figure}

The existing formal specification in Agda for the Consensus and Ledger layer \parencite{consensus-spec,cardano-formal-ledger-specs} can be naturally extended with the details of vote and certificate validation, in particular for conformance testing.

\subsection{Handling votes and certificates from the future}\label{sec:votes certs from the future}

A vote or certificate is \emph{from the future} if its round has not yet begun according to the local wallclock.
We suggest to handle such votes just like \emph{headers} from the future are already being handled:
\begin{itemize}
\item
  Upon receiving a vote or certificate $x$ from the \emph{near} future (which is currently defined to be at most \qty{2}{\s} in the future), an artifical delay is introduced such that $x$ is no longer from the future, after which $x$ is processed further.

  This rule is motivated by the insight that it would be unreasonable to expect clocks between honest nodes across the world to be perfectly synchronized.

  The artifical delay might not be necessary for Peras votes/certificates as there is, in contrast to headers/blocks, no incentive to diffuse your vote as early as possible (as long as it still arrives within a Peras round).
  However, using the exact same logic as for headers has a conceptual simplicity.
\item
  Upon receiving a vote or certificate from the \emph{far} future (defined to be more than \qty{2}{\s} in the future), we immediately disconnect from the sending peer as this constitutes adversarial behavior.
\end{itemize}

\subsection{Handling votes and certificates from the past}\label{sec:votes certs from the past}

The formal specification of Peras in~\cite{peras-cip} allows to diffuse votes (and hence processes the resulting certificates) that lie far arbitrarily in the past.
In an actual implementation, this is undesirable, as the node does not maintain the necessary state (e.g.\ arbitrarily old stake distributions) in order to even be able to validate such votes/certificates.
We now describe an approach for resolving this in an actual implementation.

\begin{description}
\item[Distant past]
  If a caught-up node observes a vote or certificate for the first time while its round is already from the distant past (i.e.\ more than \Tcp{} slots old), it can safely ignore/discard it, as such votes and certificates can only affect (and hence further strengthen compared to alternative chains) the common prefix of the honest nodes, i.e.\ the already-immutable part of their selection.

  A consequence of this rule is that two honest caught-up nodes can differ harmlessly in whether they have seen a certificate for a historical round, namely when an adversary diffuses certain votes/certificates very late, such that some honest nodes still store a particular certificate for an old (but still just young enough) round, but other nodes ignore it as it is already (barely) too old.
\item[Near past]
  In contrast, we handle votes and certificates from the near (i.e., not distant) past by disallowing votes from past rounds (with some tolerance to account for an acceptable clock skew), but still allowing to diffuse such certificates.

  This is justified by the fact that honestly cast and diffused votes have ample time to be diffused within a Peras round.
  On the other hand, an adversary can delay their votes for some round $r$ arbitrarily, for example such that some honest nodes consider them to have arrived on time for round $r$ and hence observe a quorum, while other honest nodes do not.
  However, despite not receiving these votes, the nodes in the latter category still quickly learn of the quorum in round $r$ as the certificate is still diffused separately.

  This approach is compatible in behavior with the formal specification, and directly bounds the amount of vote diffusion work that an honest client has to perform per round (namely, to download the votes of that round).
  In particular, it prevents attackers from releasing lots of votes from past rounds in a burst.
\end{description}

\subsection{Monotonicity of honestly observed certificates}\label{sec:cert monotonicity}
For future reference, we observe that an honest nodes observes Peras certificates in \emph{almost} monotonically increasing order of round numbers in the sense below.

If an honest caught-up node observes a certificate for round $r$ but has not yet done so for round $r-1$, this means that
\begin{itemize}
\item
  either a cooldown just ended, during which no honest nodes voted and so no certificate was created for many rounds, so the node has definitely seen all certificates for rounds smaller than $r$,
\item
  or other honest nodes voted in round $r$ due to them observing a certificate for round $r-1$ at that point, so it is only a matter of time until the node also receives that certificate (which we can assume to happen before round $r+1$ starts, as $\perasRoundSlots \gg \Delta$, where $\Delta$ is the maximum network delay).

  An adversary with sufficient stake $\alpha > 1-\perasQuorum$ can cause this scenario by withholding its votes during round $r-1$ until closely before the start of round $r$, and then diffusing them only to a subset of the honest nodes in time before round $r$.
  Also see \cref{sec:attack block creation rule}.
\end{itemize}
In both cases, the sequence ${(r_k)}_{k\in\mathbb{N}}$ of round numbers of certificates observed by an honest node increases \emph{almost} monotonically, i.e.\ $\max_{i<k} r_k \le  r_{k+1} - 1$ for all $k\in\mathbb{N}$.

%%% Local Variables:
%%% mode: latex
%%% TeX-engine: xetex
%%% TeX-master: "../peras-design"
%%% End:
