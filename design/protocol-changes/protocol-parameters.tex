\section{Protocol parameters}\label{sec:protocol parameters}

The Peras protocol is influenced by various new protocol parameters, also see~\cite{peras-cip}, which we list in \cref{fig:protocol parameters}.
For explicitness and to avoid confusion, we use a relatively verbose naming scheme.
The Cardano Ledger allows protocol parameters to be controlled via on-chain governance.
We indicate whether this flexibility is likely to be useful.

\begin{table}[h]
  \centering
  \begin{tabular}{c c c c c}
    \toprule
    Name & Unit & Symbol & Feasible value & Governable? \\
    \midrule
    \perasRoundSlots{} & \unit{\slot} & $U$ & \num{90} & $\checkmark$ \\
    \perasBlockMinSlots{} & \unit{\slot} & $L$ & \numrange{30}{90} & $\checkmark$ \\
    \perasBlockMaxSlots{} & \unit{\slot} & n.a. & $T_{\mathrm{CQ}} = \kcg/\asc$ & $\checkmark$ \\
    \perasCertMaxSlots{} & \unit{\slot} & $A$ & \alex[inline,inlinewidth=4cm]{see ongoing Peras CIP discussion} & ✘ \\
    \perasIgnoranceRounds{} & round & $R$ & \alex[inline,inlinewidth=4cm]{see ongoing Peras CIP discussion} & ✘ \\
    \perasCooldownRounds{} & round & $K$ & \alex[inline,inlinewidth=4cm]{see ongoing Peras CIP discussion} & ✘ \\
    \perasBoost{} & blocks & $B$ & \num{15} & $\checkmark$ \\
    \perasQuorum{} & weight & $\tau$ & $3/4 \cdot \perasN$ & ✘ \\
    \perasN{} & committee seats & $N$ & \numrange{500}{1000} & $\checkmark$ \\
    \perasVoteSizeLimit{} & \unit{\byte} & n.a. & \qty{200}{\byte} & $\checkmark$ \\
    \perasCertSizeLimit{} & \unit{\byte} & n.a. & \qty{20}{\kilo\byte} & $\checkmark$ \\
    \bottomrule
  \end{tabular}
  \caption{Peras protocol parameters}\label{fig:protocol parameters}
\end{table}

\begin{description}
\item[\perasRoundSlots]
  Peras round length, number of slots per Peras round.
\item[\perasBlockMinSlots]
  Peras block selection offset, the minimum age (in slots) of a block to be voted on at the beginning of a Peras round.

  Note that the rather small values of this parameter allow relatively weak adversary to execute \enquote{vote splitting attacks}, see \cref{sec:honest vote splitting} for more details.
\item[\perasBlockMaxSlots]
  The maximum age (in slots) of a block to be voted on at the beginning of a Peras round.

  This parameter is new compared to the CIP~\cite{peras-cip}.
  It is motivated to allow validating votes/certificates ahead of the current chain, especially while syncing.\alex{link to corresponding appendix}
\item[\perasCertMaxSlots]
  The maximum age of a certificate to be included in a block.
\item[\perasIgnoranceRounds, \perasCooldownRounds]
  Lengths of the chain ignorance period and the cooldown period.
  Determine for how long honest nodes will not vote after an unsuccessful Peras round.
\item[\perasBoost]
  The extra chain weight that a certificate gives to a block.
\item[\perasQuorum]
  The total weight of votes required to create a certificate.

  We assume that the total weight of all votes in a Peras round is at most \perasN{}.
\item[\perasN]
  The expected number of committee seats, i.e.\ the total number of votes that can be cast per round.
\item[\perasVoteSizeLimit, \perasCertSizeLimit]
  Maximum size (in bytes) of Peras votes and certificates.
\end{description}

\alex{Double-check $k_{\mathrm{Peras}}$}

For a more detailed discussion of the interactions of the various parameters we refer to~\cite{peras-cip}.

%%% Local Variables:
%%% mode: latex
%%% TeX-engine: xetex
%%% TeX-master: "../peras-design"
%%% End:
