\section{Vote and certificate diffusion}%
\label{sec:certificate-vote-diffusion}

This section presents our proposed protocols for diffusion of votes (\cref{sec:vote-diffusion}) and certificates (\cref{sec:certificate-diffusion}) between peers, based on the object diffusion mini-protocol described in \cref{sec:object-diffusion-protocol}.
Finally, \cref{sec:vote cert alternatives} presents a discussion about alternative designs that we have considered.

\subsection{Vote diffusion mini-protocol}%
\label{sec:vote-diffusion}

\paragraph{Requirements.}
Diffusion of votes takes place between caught-up nodes in every Peras round.
The core requirement is that all honest pools observe a quorum before the end of a round, as long as votes with weight of at least \perasQuorum{} have been cast for the same block in that round.

Concretely, assuming $\perasN = 1000$ and a vote size of $\qty{164}{\byte}$ (\cref{fig:protocol parameters,fig:vote cert metrics}), the total size of the votes of a round is $\perasN \cdot \qty{164}{\byte} = \qty{164}{\kilo\byte}$, which needs to be diffused within $\perasRoundSlots = 90$ slots/seconds.
This is a more relaxed timeliness constraint compared to block diffusion, where blocks of size up to $\qty{90}{\kilo\byte}$ need to be diffused within $\lesssim \Delta = 5$ slots/seconds.

\paragraph{Using object diffusion.}

We propose to use the generic object diffusion mini-protocol (\cref{sec:object-diffusion-protocol}) instantiated for votes.
In particular, we want:
%
\begin{description}
\item[\argfont{object}]
  A Peras vote.
\item[\argfont{id}]
  A vote ID, i.e.\ a pair of a round number and an identifier for a committee seat of that round, determining the pool identity.

  Concretely, in scheme described in~\cite{peras-cert-report}, this is either the hash of the cold key of the voting pool, or (as an optimization) an index into the stake distribution.
\item[\argfont{objectIds}, \argfont{responseToIds}]
  A list of \argfont{id}s. Correspondingly, \argfont{objectIds} is the identity function.
\item[\argfont{initialPayload}]
  No payload is necessary; only votes near the current round of the system are ever diffused.
\end{description}

The server advertises the vote IDs of the votes it has received for the current round (sorted by weight in decreasing order), as well as (with lower priority) vote IDs for unsuccessful rounds of the near past, cf.~\cref{sec:votes certs from the past}.
(As honest nodes stop voting after an unsuccessful round, all later votes during a cooldown periods have been cast by the adversary.)
Additionally, it serves the votes corresponding to the unacknowledged vote IDs.

We now describe the dynamics of an honest node engaging as the client in vote diffusion with its peers, assuming that it is caught-up.
\begin{itemize}
\item
  The node continually requests more vote IDs from its peers (with a limit on unacknowledged votes), using non-blocking or blocking requests depending on whether there are are outstanding unacknowledged votes.
\item
  At the beginning of a Peras round, the node will start receiving lots of new vote IDs for that round from all of its (honest) peers.
  For each such vote ID, the node will ask one peer (or a small number in parallel) out of the peers that offered this vote ID for the corresponding vote, relying on protocol-level timeouts for a prompt delivery (or otherwise disconnecting from the peer).

  By limiting the number of votes that are being downloaded in parallel, the traffic is implicitly spread out over the first few seconds of the round, bounding the spike of network activity.

  This strategy naturally handles adversarial vote equivocation, cf.~\cref{sec:attack equivocations}.
\item
  The client will \emph{not} download the vote corresponding to a vote ID if its round either lies in the distant past (\cref{sec:votes certs from the past}), or in the near past for which the node has already observed a successful quorum.
\item
  It might seem appealing to also stop downloading votes for the current round as soon as we have observed a quorum (via votes of weight \perasQuorum{}).

  However, then the node can not relay these votes, and a downstream peer might be missing exactly some of these votes to observe a quorum of votes by itself.
  Our design still handles this case directly by also running certificate diffusion between caught-up nodes (\cref{sec:certificate-diffusion}), but downloading and validating only a few votes is more efficient than transferring a certificate.\footnote{
  It is conceivable that in a variant of~\cite{peras-cert-report}, certificates will be only slightly larger than votes, in which case the difference in efficiency might be negligible.}
A compromise would be to stop downloading votes for the current round as soon as votes of weight larger than $\sigma$ have been observed, where $\perasQuorum < \sigma < 1$.
\item
  The client always checks that the votes indeed correspond to the advertised vote IDs, and that the votes are in fact valid, and disconnects from the offending peer otherwise.
  Together with the fact that the number of votes per round is bounded by the size of the Peras committe, this bounds the work of the client.
\end{itemize}
The node does not send requests for vote IDs or votes while it is syncing (in particular, it could not even validate them at this point), which can be determined by the Genesis State Machine \parencite{genesis-implementation-documentation} or a more ad-hoc criterion like the proximity of the chain tip to the current wall-clock time.

This protocol shares several similarities with Tx-Submission, and we anticipate that the implementation can benefit from the insights of the design in \texttt{cardano-node}.
In particular, the IOE Network team has been reworking the inbound side of Tx-Submission\footnote{\url{https://github.com/IntersectMBO/ouroboros-network/pull/4887}} to more efficiently download transactions from different peers (avoiding repeated downloads).

\subsection{Certificate diffusion mini-protocol}%
\label{sec:certificate-diffusion}

\paragraph{Requirements.}

Diffusion of certificates is required both for syncing (\ref{enumi:cert diffusion:syncing}) and caught-up (\ref{enumi:cert diffusion:caught up}) nodes:

\begin{enumerate}
\item\label{enumi:cert diffusion:syncing}
  Primarily, syncing nodes/pools need to be able to obtain historical certificates in order to choose the correct historical chain, even though certificates do not (meaningfully) affect the ledger state.

  Concretely, nodes bootstrapping via Ouroboros Genesis (to minimize trust assumptions) need to be able to assess the weight of competing historical chains in order to resist adversarial \emph{long-range attacks} \parencite{genesis-implementation-documentation}.
  Computing the weight of a chain requires the corresponding certificates indicating which blocks are boosted.
  See \cref{sec:weighted genesis} for more details.
  Even if a syncing node did not encounter any adversarial challengers (and hence didn't ever need to compute weight), it still must get all certificates. Indeed, \emph{other} peers syncing from that node (e.g., in the future) might encounter adversaries and therefore require those certificates.
  Briefly, there is a \enquote{duty to remember} certificates even if they are not directly useful for the caught-up node anymore.

  The Cardano implementation of Ouroboros Genesis \parencite{genesis-implementation-documentation} requires that syncing nodes are always connected to at least one honest peer.
  (This requirement is called the \enquote{honest availability assumption}.)
  In practice, this is guaranteed by connecting to a sufficient number of appropriately sampled peers.
  To reduce the load on these peers (and for a more efficient use of resources generally), downloading the same certificates from multiple peers must be avoided.%
%
\footnote{We find this motivation in other parts of Ouroboros. For instance, as part of the Cardano Genesis implementation, we ensured that both block bodies and headers were downloaded only once.\cite{genesis-implementation-documentation}}
\item\label{enumi:cert diffusion:caught up}
  Additionally, diffusing certificates even between caught-up nodes allows mitigating vote-equivocation attacks, cf.~\cref{sec:attack equivocations}.
\end{enumerate}

We note that certificates can also be diffused in blocks (in order to coordinate the end of a cooldown); this is orthogonal to the discussion in this section.

\paragraph{Using object diffusion.}

We propose to again use the generic object diffusion mini-protocol (\cref{sec:object-diffusion-protocol}) instantiated for certificates.
In particular, we want:

\begin{description}
\item [\argfont{object}]
  A certificate.
\item [\argfont{id}]
  A Peras round number.
\item [\argfont{objectIds}, \argfont{responseToIds}]
  In its simplest form, a list of Peras round numbers, and the identity function.

  As an optimization, compact/compressed representations are possible, for example
  \begin{itemize}
  \item
    a round followed by a bitset for the subsequent rounds, indicating whether a certificate is present, or
  \item
    a round followed by a run-length encoding of the subsequent rounds.
    This is motivated by the observation that in Peras, rounds are (not) successful in larger runs, either because all rounds are successful due to sufficient honest votes, and if not, a lengthy cooldown period (with no successful rounds to due honest abstention) of unsuccessful rounds.
  \end{itemize}
  The \argfont{responseToIds} function is then decoding this compact representation.
\item [\argfont{initialPayload}]
  A Peras round number.

  Partially synced/recently caught-up peers can use this to receive certificates starting from the first round for which they do not yet have a certificate, avoiding the transfer of older, already-downloaded data.\footnote{
  This is conceptually analogous to \msg{MsgFindIntersect} in the ChainSync protocol.}
\end{description}

An honest server answers requests for more round numbers by sending those for which it has a certificate in ascending order, until the client is caught-up, in which case the server blocks (or returns empty results) until it observes new certificates arising from successful Peras voting rounds.%
\footnote{The resulting sequence of round numbers is \emph{almost} monotonically increasing, with a possible exception once the client has (almost) caught-up, see below.}
At any time, it will serve the certificates corresponding to unacknowledged round numbers, while enforcing an upper bound on the this quantity.

This design leverages the fact that there can only be at most one certificate per round, which allows us to elide data like the point of the block that is being certified/boosted, justifying the use of round numbers as certificate IDs.
In particular, it does not matter who we download a certificate for a particular round from.

We now give a high-level description of the envisioned dynamics of this protocol.

\begin{itemize}
\item
  Consider a syncing node via Ouroboros Genesis which still needs to catch up a significant part of the historical chain.
  The node continually requests sizeable chunks of round numbers via \msg{MsgRequestObjIdsNonBlocking} for each peers.
  For every round number that is advertised by at least one peer, the corresponding certificates are downloaded from the peers via \msg{MsgRequestObjs} (as a first step, a singular designated peer which a simple batching strategy could be used; but more sophisticated strategies, similar to the existing BlockFetch logic are conceivable).
  The client uses protocol-level pipelining to avoid round-trip delays and to make full use of the available bandwidth.
\item
  As the node is eventually done catching up, its peers run out of certificates to serve, and the node will start sending \msg{MsgRequestObjIdsBlocking} instead.\footnote{
  In particular, an incomplete/empty reply to \msg{MsgRequestObjIdsBlocking} is analogous to \msg{MsgAwaitReply} in the ChainSync protocol.}
  Usually, it will receive one new round number per peer every \perasRoundSlots{} slots, but given that the node now participates in vote diffusion, it is not necessary to actually download the certificate.

  An exception is the scenario where the node has downloaded all votes for a round without observing a quorum, but still received this round number via the certificate diffusion protocol.
  In this case, the node will download the certificate via certificate diffusion.
  This can only happen due to adversarial vote equivocation, see \cref{sec:attack equivocations}.
\end{itemize}

Clients can ensure progress in this protocol w.r.t.\ adversarial servers by enforcing appropriate timeouts and a monotonicity property on the advertised round numbers:
\begin{enumerate}
\item\label{enumi:cert diffusion timeouts}
  Prompt delivery of requested certificates can be ensured by protocol-level timeouts, or via a more elaborate mechanism like a \emph{leaky token bucket} as used in the Cardano Genesis implementation  \parencite{genesis-implementation-documentation} which handles temporary latency spikes more gracefully.
\item\label{enumi:cert diffusion monotonicity}
  The sequence of round numbers sent by the server must increase \emph{almost} monotonically, as honest nodes (acting as servers) can observe the certificate for round $r$ before the one for round $r-1$ in edge cases, see~\cref{sec:votes certs from the past}.

  This almost-monotonicity requirement can be enforced implicitly by an appropriate leaky token bucket (also see~\ref{enumi:cert diffusion timeouts}) via an approach analogous to the \enquote{Limit on Patience}, the mechanism used in the Cardano Genesis implementation \parencite{genesis-implementation-documentation} to guarantee progress in ChainSync while syncing.

  In short, the idea is to make sure that the server sends round numbers higher than anything it has sent before at a minimum rate on average, or that the server advertises that it has no more certificates.
  An honest server will have no trouble in doing that, as it will only ever send round numbers in non-monotonic order when the client is almost caught-up.
  Once it is caught-up, this mechanism can be disabled, just like the Limit on Patience.
\end{enumerate}

Finally, we note that in order to conclude that a round \emph{does} have a corresponding certificate, it is enough to download such a cert from any peer, while concluding that there is \emph{no} certificate for a round $r$ requires information from \emph{all} peers (in the form of advertising a certificate for a round sufficiently larger than $r$ as per~\ref{enumi:cert diffusion monotonicity}, or by the peer indicating that they have no more certificates at the moment).

\subsection{Alternatives}\label{sec:vote cert alternatives}

We briefly discuss two alternatives to the design of vote and certificate diffusion presented above.
\begin{itemize}
\item
  One could combine vote and certificate diffusion into one custom protocol.
  This would allow for some optimizations, such as responding with a certificate when a client asks for a vote that it subsumes, and could make certain interactions between the vote and certificate clients explicit.

  However, the separation of votes and certificates actually allows the individual protocols to be simpler and more focused, and the aforementioned optimization does not seem relevant in practice, in particular as one still needs synchronization between the clients of different peers.

  Also, there is an advantage in reusing object-diffusion mini-protocol (\cref{sec:object-diffusion-protocol}) due to its similarity to the existing Tx-Submission protocol, and the planned use for Mithril \parencite{dmq-cip} and Ouroboros Leios.
\item
  Certificate diffusion could use \emph{two} protocols similar to how chains are diffused in Cardano.
  Concretely, the first protocol would be similar to ChainSync and diffuse just certificate IDs (i.e.\ round numbers) and the second protocol would be an instantiation of BlockFetch with certificate IDs instead of block points and certificates instead of blocks.

  However, this approach seems overly complicated for the purpose of certificate diffusion.
  The primary motivation for having separate ChainSync and BlockFetch protocls is the \emph{header-body split} \parencite{shelley-data-diffusion-networking}, and ChainSync is specifically designed for stateful chain following, whereas certificates do not have an inherent chain structure (despite voting for blocks on a block chain).
\end{itemize}

%%% Local Variables:
%%% mode: latex
%%% TeX-engine: xetex
%%% TeX-master: "../peras-design"
%%% End:
