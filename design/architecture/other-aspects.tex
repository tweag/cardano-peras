\section{Additional considerations}

\subsection{Node-to-client protocols}

In order to make use of Peras and its improved settlement guarantees under optimistic conditions, applications need to monitor whether or how many of the block containing their transaction of interest and its descendants are boosted via a Peras certificate.

For this, we propose to run certificate diffusion (\cref{sec:certificate-diffusion}) also as a node-to-client protocol, such that clients can process these themselves for their use case.
For example, such a separate service can provide an easy-to-use API that can answer queries about recent blocks with concrete settlement probabilities based on the observed certificates.

Potentially, one could add a simple protocol that reports the weight on top of a (volatile) block, as the node already maintains the data necessary for this functionality.

We do not see a strong use case for exposing votes as a node-to-client protocol; monitoring tools for Peras can likely use the node-to-node protocol directly.

\subsection{Lighter chain syncing and following}

Peras enables more lightweight chain syncing and following under optimistic conditions, i.e.\ when Peras is not in a cooldown period.
Observing a valid Peras certificate boosting a block $B$ implies in particular that at least one honest node fully validated and selected $B$.
Therefore, when adopting $B$ or any of its predecessors, it is sound to skip cryptographic checks like signature verification as well as script execution, as these do not influence the resulting ledger state, saving the bulk of the CPU work necessary to fully validate them.

Here, Peras certificate play the same role as endorsement certificates in Leios, see~\cite[Section 4.4.2]{leios-design-goals-concepts}.

\subsection{Key registration period}

Votes and certificates will require pool operators to register new cryptographic keys, cf.\ \cref{sec:realizing votes certs}.
As Peras is ineffective (due to its cooldown periods) until a significant number of SPOs have done so, we propose to only enable Peras after another (intra-era) hard fork.
However, nodes can already create (if the pool owner registered a key on chain) and relay votes during this period, without the resulting certificates (if any) actually taking any effect.
This allows to monitor the dynamics and various statistics of vote diffusion in the \enquote{real world} before Peras goes fully live.

\subsection{KES agent integration}

Votes and certificates are likely going to use key-evolving signatures underneath, cf. \cref{sec:crypto votes agg} for improved resilience against long-range attacks (cf.\ \cite{david2018ouroboros} and \cref{sec:storing historical certs}), just like the existing KES scheme used for block signing.
Soon, the Haskell cardano-node will support a KES agent\footnote{\url{https://github.com/input-output-hk/kes-agent}} to actually deliver on the promise that an attacker who corrupts a node does not get access to past signing keys.
Naturally, it would therefore be desirable to let the KES key used by Peras also be handled by the KES agent, see \url{https://github.com/input-output-hk/kes-agent/issues/52}.

%%% Local Variables:
%%% mode: latex
%%% TeX-engine: xetex
%%% TeX-master: "../peras-design"
%%% End:
