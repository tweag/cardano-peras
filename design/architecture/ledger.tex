\section{Changes to the Ledger}\label{sec:ledger changes}

The Cardano Ledger \parencite{shelley-ledger-specs,cardano-formal-ledger-specs} needs minor modifications to accomodate Peras.

Peras requires new protocol parameters, see \cref{sec:protocol parameters}, which should be a routine change.

The cryptographic scheme for votes and certificates \parencite{peras-cert-report} will likely require pools to register new keys, which implies corresponding changes to pool (re-)registration.\footnote{
  This also raises other questions like how to handle pools that did not register these new keys as Peras gets enabled.
  A natural approach is to monitor the chain until pools of sufficient stake have done so, and only then enable Peras, ignoring the remaining pools until they have registered appropriate keys on chain.}

Apart from that, the Ledger needs to be modified to account for Peras certificates which are included on chain to coordinate the end of a cooldown period.
We note that certificate are morally part of the Consensus layer (in particular, validating them requires the epoch nonce, which no other ledger rule does); however, certificates are likely too large to be stored in a header.

Concretely, we propose the following checks for a block body $B$ in round $r$ containing a certificate \cert{}:
\begin{enumerate}
\item
  The certificate \cert{} must be valid.
\item
  The round of \cert{} must be strictly greater than the round of the previous certificate on chain, and it must lie between $r-\perasCertMaxSlots$ and $r$, i.e.\
  \[ r-\perasCertMaxSlots \le \round(\cert) \le r \;. \]
  This ensures that \cert{} is neither too old nor too new, such that the ledger state that $B$ is validated against contains the necessary information to validate $r$.
\item
  The size of the certificate must be bounded.
  Concretely, we propose a protocol parameter \perasCertSizeLimit{} for the size of a certificate, and additionally, we let the size of the certificate count towards the maximum block body size.
\end{enumerate}
The only change to the ledger state is the addition of the round number of the last certificate on chain, and the protocol parameters and stake distribution for the previous epoch in case \cert{} is in a previous epoch compared to $B$.%
\footnote{The ledger state already keeps this data of the previous epoch around for its reward calculation.}

\subsection{Analysis}
Note that these rules allow the adversary to needlessly include certificates in chain even if there is no current/upcoming cooldown period.
This does not impact the purpose of on-chain certificates:
If the system does enter a cooldown (via honest nodes stopping to vote), the adversary will run out of certificates to include on chain due to the monotonicity property of round numbers.

Moreover, this monotonicity property also ensures that the adversary can (on average) only include one certificate per Peras round.
\begin{itemize}
\item
  This means that, on average (assuming no cooldown periods), an attacker with sufficient stake to be elected at least once per round on average\footnote{Concretely, this is satisfied for an adversary with stake $\alpha$ if $\phi(\alpha)\cdot\perasRoundSlots \ge 1$, so for $\alpha \ge 21.8\%$ if $f=1/20$ and $\perasRoundSlots=90$.} can include
  \[\frac \perasCertSizeLimit \perasRoundSlots\,\unit{B/\slot}\]
  without anyone paying for the associated cost of bandwidth/storage.

  For comparison, the fee for including $\perasCertSizeLimit$ bytes on the chain as part of a transaction is given by $\mi{minfeeA}\cdot \perasCertSizeLimit$.
  For realistic values on the conservative end (using $\mi{minfeeA} = 44$ and $\mi{minfeeB} = 155381$ on Cardano mainnet as of epoch 537\footnote{See e.g.\ \url{https://cexplorer.io/params}.}, $\perasRoundSlots=90$ and $\perasCertSizeLimit = \SI{20}{kB}$, cf.~\cref{fig:vote cert metrics}), this would correspond to a cost of
  \[ \frac{\mi{minfeeA\cdot \perasCertSizeLimit + \mi{minfeeB}}}{\perasRoundSlots} \approx \SI{11504}{\lovelace/\slot} \approx \SI{41.42}{\ADA/h} \]
  assuming a slot length of one second.

  For context, note that implementing Peras also entails diffusing \perasN{} votes per round (and the total size of the votes is significantly large than \perasCertSizeLimit{}), and the bandwidth induced by this is also currently not planned to be compensated/incentivized.
\item
  Additionally, we need to validate these uselessly included certificates.
  As they do not affect the ledger state, this validation can in principle be performed in parallel to the ledger rules, partially mitigating the impact.
\end{itemize}
Finally, note that otherwise-honest nodes have no incentive to include such useless certificates, in particular as they usually could instead include more transactions (which pay fees) in their blocks.

\subsection{Alternatives}
If this cost is considered to be unacceptable, a drastic measure would be to require pools to pay a comparable fee when they include a certificate in a block (which is a rare thing for honest nodes).
However, such a fee for pools is unprecedented, and raises various unresolved questions regarding incentives and what to do when honest pools run out of funds to pay this new fee.

Another option would be to modify the Peras protocol to reliably determine whether an honest node could have included a certificate into a particular block.
However, we are not aware of an easy way to do this.


%%% Local Variables:
%%% mode: latex
%%% TeX-engine: xetex
%%% TeX-master: "../peras-design"
%%% End:
