\section{Protocol parameters}\label{sec:protocol parameters}

For context, we list existing (Praos/Genesis) protocol parameters in~\cref{fig:praos protocol parameters}.

\begin{table}[h]
  \centering
  \begin{tabular}{c c c c}
    \toprule
    Description & Unit & Symbol & Mainnet value \\
    \midrule
    Active slot coefficient & \unit{\slot^{-1}} & \asc{} & $1/20$ \\
    Security parameter for chain growth & block & \kcp{} & \num{2160} \\
    \makecell{Chain growth window size\\{\small to achieve common prefix}} & \unit{\slot} & \Tcp{} & $3\cdot\kcp / \asc = \num{129600}$ \\
    \makecell{Chain quality window size\\{\small to guarantee at least one honest block}} & \unit{\slot} & \Tcq{} & $\kcp/\asc = \num{43200}$ \\
    Genesis window & \unit{\slot} & \sgen & \Tcp{} \\
    \bottomrule
  \end{tabular}
  \caption{Existing Praos/Genesis protocol parameters}\label{fig:praos protocol parameters}
\end{table}

The Peras protocol is influenced by various new protocol parameters, also see~\cite{peras-cip}, which we list in \cref{fig:protocol parameters}.
The values of these parameters are not yet final, and justifying their values (especially the ones related to cooldowns) is out of scope for this document.
For explicitness and to avoid confusion, we use a relatively verbose naming scheme.
The Cardano Ledger allows protocol parameters to be controlled via on-chain governance.
We indicate whether this flexibility is likely to be useful.

\begin{table}[h]
  \centering
  \begin{tabular}{c c c c c}
    \toprule
    Name & Unit & Symbol & Feasible value & Governable? \\
    \midrule
    \perasRoundSlots{} & \unit{\slot} & $U$ & \num{90} & $\checkmark$ \\
    \perasBlockMinSlots{} & \unit{\slot} & $L$ & \numrange{30}{900} & $\checkmark$ \\
    \perasBlockMaxSlots{} & \unit{\slot} & n.a. & $T_{\mathrm{CQ}} = \kcp/\asc$ & $\checkmark$ \\
    \perasHealingSlots{} & \unit{\slot} & \Theal & t.b.d. & ✘ \\
    \perasIgnoranceRounds{} & round & $R$ & $\lceil (\Theal + \Tcq)/U \rceil$ & ✘ \\
    \perasCooldownRounds{} & round & $K$ & $\lceil (\Theal + \Tcq + \Tcp)/U \rceil + 1$ & ✘ \\
    \perasCertMaxRounds{} & round & $A$ & \perasIgnoranceRounds{} & ✘ \\
    \perasBoost{} & block & $B$ & \num{15} & $\checkmark$ \\
    \perasQuorum{} & weight & $\tau$ & $3/4$ & ✘ \\
    \perasN{} & committee seat & $N$ & \numrange{500}{1000} & $\checkmark$ \\
    \perasVoteSizeLimit{} & \unit{\byte} & n.a. & \qty{200}{\byte} & $\checkmark$ \\
    \perasCertSizeLimit{} & \unit{\byte} & n.a. & \qty{20}{\kilo\byte} & $\checkmark$ \\
    \bottomrule
  \end{tabular}
  \caption{New Peras protocol parameters}\label{fig:protocol parameters}
\end{table}

\begin{description}
\item[\perasRoundSlots]
  Peras round length, number of slots per Peras round.
\item[\perasBlockMinSlots]
  Peras block selection offset, the minimum age (in slots) of a block to be voted on at the beginning of a Peras round.

  Note that the rather small values of this parameter allow relatively weak adversary (i.e. $\le \qty{25}{\percent}$ stake) to execute \enquote{vote splitting attacks} that can force Peras into a cooldown period, see \cref{sec:honest vote splitting} for more details.
  Larger values for \perasBlockMinSlots{} make such adversarially-induced cooldowns less likely, but they introduce additional settlement latency even in the optimistic case.
\item[\perasBlockMaxSlots]
  The maximum age (in slots) of a block to be voted on at the beginning of a Peras round.

  This parameter is new compared to the CIP~\cite{peras-cip}.
  It is motivated to allow validating votes/certificates ahead of the current chain, especially while syncing.\footnote{
  We note that the exact details of forecasting are still subject to discussions, in particular pending feedback by the Peras research team, cf.~\url{https://github.com/tweag/cardano-peras/issues/25}.}
\item[\perasHealingSlots]
  The amount of slots needed to heal from a failed Peras voting round, depending on $\perasBoost$.
  A concrete value has yet to be chosen.
\item[\perasIgnoranceRounds, \perasCooldownRounds]
  Lengths of the chain ignorance period and the cooldown period.
  Determine for how long honest nodes will not vote after an unsuccessful Peras round.

  We note that \Theal{} needs to be parameterized for a very strong adversary (in order to make up for a late boost/certificate that the adversary might have up their sleeve).
  In contrast, \Tcq{} and \Tcp{} can be parameterized for weaker adversaries (i.e.~\qty{25}{\percent} stake or less), as they are only used to let honest nodes restart voting at the same time, which is not critical for safety.
  See \cref{sec:attack cooldowns} for more details.
\item[\perasCertMaxRounds]
  The maximum age of a certificate to be included in a block.
\item[\perasBoost]
  The extra chain weight that a certificate gives to a block.
\item[\perasQuorum]
  The total weight of votes required to create a certificate, assuming that the total (expected) weight of a committe is $1$.
\item[\perasN]
  The expected number of committee seats, i.e.\ the total number of votes that can be cast per round.
\item[\perasVoteSizeLimit, \perasCertSizeLimit]
  Maximum size (in bytes) of Peras votes and certificates.
\item[\kcp]
  The security parameter (reinterpreted to measure weight of chains instead of just blocks) needs to be increased to retain the same security as in pure Praos.
\end{description}

For a more detailed discussion of the interactions of the various parameters we refer to~\cite{peras-cip}.

%%% Local Variables:
%%% mode: latex
%%% TeX-engine: xetex
%%% TeX-master: "../peras-design"
%%% End:
