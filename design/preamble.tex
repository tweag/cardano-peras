\documentclass[10pt]{twaudit}
%% The draft option disables microtype and displays overful hboxes in the
% text, plus it's faster to compile

\usepackage{enumitem}
\setlist[enumerate, 2]{label = \roman*.}

\usepackage{csquotes}

\usepackage{siunitx}
\DeclareSIUnit{\slot}{\text{slot}}
\DeclareSIUnit{\lovelace}{\text{lovelace}}
\DeclareSIUnit{\ADA}{\text{ADA}}

%% for llparenthesis and rrparenthesis
\usepackage{stmaryrd}

\usepackage{verbatim}

\usepackage{multirow}

\usepackage{makecell}

%% For figures that wrap around text instead of floating around
\usepackage{wrapfig}

%% For drawings
\usepackage{tikz}
\usetikzlibrary{decorations.pathreplacing, shapes}
\usepackage{xcolor}

\usepackage{tcolorbox}

%% For in-document notes between ourselves
\usepackage[textwidth=8em, obeyFinal]{todonotes}
\newcommand\niols[2][]{\todo[color=red!30, #1]{#2 [Niols]}}
\newcommand\alex[2][]{\todo[color=blue!30, #1]{#2 [Alex]}}
\newcommand\nbacquey[2][]{\todo[color=purple!30, #1]{#2 [N. Bacquey]}}

%% Since not all reports might contain code, we also bring minted separately.
\usepackage{minted}
\usemintedstyle{bw}
\setminted{fontsize=\footnotesize}
\setmintedinline{fontsize=\normalsize}
\newcommand{\hs}[1]{\mintinline{haskell}{#1}}
\newcommand{\txt}[1]{\mintinline{text}{#1}}

%% Use this for variable names in math mode longer than 1 character
\newcommand{\mi}[1]{\mathtt{#1}}

\newcommand{\asc}{\ensuremath{\mi{asc}}}
\newcommand{\kcp}{\ensuremath{\mi{kcp}}}
\newcommand{\sgen}{\ensuremath{s_{\mi{gen}}}}
\newcommand{\Tcp}{\ensuremath{T_{\mi{cp}}}}
\newcommand{\Tcq}{\ensuremath{T_{\mi{cq}}}}
\newcommand{\Theal}{\ensuremath{T_{\mi{heal}}}}

\newcommand{\perasRoundSlots}{\ensuremath{\mi{perasRoundSlots}}} % also known as U
\newcommand{\perasBlockMinSlots}{\ensuremath{\mi{perasBlockMinSlots}}} % also known as L / "block selection offset"
\newcommand{\perasBlockMaxSlots}{\ensuremath{\mi{perasBlockMaxSlots}}}
\newcommand{\perasCertMaxSlots}{\ensuremath{\mi{perasCertMaxSlots}}} % also known as A / "certificate expiration"
\newcommand{\perasHealingSlots}{\ensuremath{\mi{perasHealingSlots}}} % also known as T_heal
\newcommand{\perasIgnoranceRounds}{\ensuremath{\mi{perasIgnoranceRounds}}} % also known as R
\newcommand{\perasCooldownRounds}{\ensuremath{\mi{perasCooldownRounds}}} % also known as K
\newcommand{\perasBoost}{\ensuremath{\mi{perasBoost}}} % also known as B
\newcommand{\perasQuorum}{\ensuremath{\mi{perasQuorum}}} % also known as τ
\newcommand{\perasN}{\ensuremath{\mi{perasN}}} % also known as N / committee size
\newcommand{\perasDelta}{\ensuremath{\mi{perasDelta}}}

\newcommand{\perasVoteSizeLimit}{\ensuremath{\mi{perasVoteSizeLimit}}}
\newcommand{\perasCertSizeLimit}{\ensuremath{\mi{perasCertSizeLimit}}}

\newcommand{\cert}{\ensuremath{\mi{cert}}}
\DeclareMathOperator{\round}{round}

%% A common macro for consistent style of all properties
\newcommand\property[1]{\textsc{#1}}

%% Praos/Genesis properties
\newcommand\praosCommonPrefixName{\property{Praos Common Prefix}}
\newcommand\chainGrowthName{\property{Chain Growth}}
\newcommand\limitOfPatienceName{\property{Limit of Patience}}
\newcommand\lengthOfCompetingChainsName{\property{Length of Competing Chains}}
\newcommand\densityOfCompetingChainsName{\property{Density of Competing Chains}}
\newcommand\densityOfCompetingChains{\hyperref[property:density-of-competing-chains]\densityOfCompetingChainsName}
\newcommand\weightedDensityOfCompetingChainsName{\property{Weighted Density of Competing Chains}}
\newcommand\weightedDensityOfCompetingChains{\hyperref[property:weighted-density-of-competing-chains]\weightedDensityOfCompetingChainsName}
\newcommand\honestAvailabilityName{\property{Honest Availability}}

%% A hack to fix the rendering of \Longrightarrow
\renewcommand\Longrightarrow{%
 \mathrel{%
  \mbox{\fontfamily{cmr}\fontencoding{OT1}\selectfont=}}%
 \joinrel\Rightarrow}

%% What's the client name? Make sure to always call with empty list of arguments
% to perserve spacing.
\TWASetClient{IOG}
\TWASetProduct{Peras}
\TWASetDate{WIP \today}
\title{Core \TWAProductOrClient{} Design}
\author{Modus Create Peras Team}

% Configure texttt with hyphenation
%
% hyphenation seems to be disabled for words with underscores or hyphens though.
%
% It is also disabled in the arguments of the \paragraph command.
\DeclareTextFontCommand{\mytexttt}{\ttfamily\hyphenchar\font=45\relax}

\hyphenation{mer-kle-i-sa-tion}

\setcounter{tocdepth}{1}

\usepackage[backend=biber,style=alphabetic,maxbibnames=99]{biblatex}
\addbibresource{bib.bib}

\usepackage{listings}
