\chapter{Sketch of a reward system for voting}\label{chap:rewards}

This section contains preliminary thoughts on how to design a reward system for Peras voting.
This is a highly complicated endeavor in mechanism design, see for example \cite{zhang2024max} for issues in Ethereum's incentive system for attestations, so this is only intended to be a starting point.

\medskip
First, we note that there are incentives for SPOs to honestly follow the Peras protocol rules even without an explicit monetary reward system.

Concretely, the stronger finality guarantees provided by Peras are themselves an incentive for the pool operators as a whole to make sure that Peras is working well (and not in an eternal cooldown period due to low participation).
However, the benefits of faster finality are weak in the beginning (when few applications such as Layer 2's already make use of Peras), resulting in a chicken-and-egg problem.
Furthermore, this kind of incentive obviously has a chance to suffer from \emph{tragedy of the commons}-like problems, where a single operator might stop participating in Peras without any noticeable effect, until many operators follow suit and participation becomes too low.

More directly, Peras voting is likely going to be \emph{accountable}, meaning that anyone, in particular people delegating their stake, can check whether/what a particular pool is voting for.
This can lead to both social pressure, as well as re-delegation of stake if a delegator is unhappy with the behavior of their pool with respect to Peras.
However, this is far from ideal, as it requires significant technical knowledge and vigilance on the side of delegators.

Also, there is the question whether the marginal additional cost of participating in Peras is actually very large.
There are only modest increases in bandwidth and storage requirements, which will likely stay below what is included in fixed-price plans by \enquote{discount} providers, so the marginal cost might well be zero for a significant fraction of the pool operators.
If not, further research is needed to determine the absolute magnitude of rewards necessary to offset these costs.

\medskip
Let us now sketch a basic reward mechanism for Peras voting.

We note that the vote and certificate scheme must provide fair eligibility and accountability (\cref{sec:cert reqs}), and we require on-chain track record of who participated in Peras voting.

If a round gives rise to a quorum, then the resulting certificate can be included on chain, and everyone voting for the correct block gets rewarded (proportional to their stake).
Otherwise, pseudo-certificates (which do not reach weight \perasQuorum{}, but still are above a given threshold) may be included, resulting in (smaller) rewards for the respective signers.
A restrictive design option would be to only allow pseudo-certificates if they are targeting a predecessor of the block containing them, making it less appealing for adversaries to vote for alternative forks.
However, this might allow adversaries who strategically diffuse forks, cf.\ \cref{sec:honest vote splitting}, to prevent certain honest pools from receiving rewards.

It seems difficult to reliably enforce pools to include certificates on chain as part of the ledger rules.
As an alternative, including (pseudo-)certificates in a block could be itself incentivized, with the minting pool creating rewards for including evidence of vote participation for as many pools/past rounds as possible.

Generally, due to block space constraints and low chain quality under high adversarial activity, it can probably not be guaranteed that \emph{all} certificates of successful rounds end up on chain.
This problem might be alleviated by the higher throughput of Leios (\cref{sec:leios}).

%%% Local Variables:
%%% mode: latex
%%% TeX-engine: xetex
%%% TeX-master: "../peras-design"
%%% End:
